%% Generated by Sphinx.
\def\sphinxdocclass{report}
\documentclass[letterpaper,10pt,english]{sphinxmanual}
\ifdefined\pdfpxdimen
   \let\sphinxpxdimen\pdfpxdimen\else\newdimen\sphinxpxdimen
\fi \sphinxpxdimen=.75bp\relax

\usepackage[utf8]{inputenc}
\ifdefined\DeclareUnicodeCharacter
 \ifdefined\DeclareUnicodeCharacterAsOptional
  \DeclareUnicodeCharacter{"00A0}{\nobreakspace}
  \DeclareUnicodeCharacter{"2500}{\sphinxunichar{2500}}
  \DeclareUnicodeCharacter{"2502}{\sphinxunichar{2502}}
  \DeclareUnicodeCharacter{"2514}{\sphinxunichar{2514}}
  \DeclareUnicodeCharacter{"251C}{\sphinxunichar{251C}}
  \DeclareUnicodeCharacter{"2572}{\textbackslash}
 \else
  \DeclareUnicodeCharacter{00A0}{\nobreakspace}
  \DeclareUnicodeCharacter{2500}{\sphinxunichar{2500}}
  \DeclareUnicodeCharacter{2502}{\sphinxunichar{2502}}
  \DeclareUnicodeCharacter{2514}{\sphinxunichar{2514}}
  \DeclareUnicodeCharacter{251C}{\sphinxunichar{251C}}
  \DeclareUnicodeCharacter{2572}{\textbackslash}
 \fi
\fi
\usepackage{cmap}
\usepackage[T1]{fontenc}
\usepackage{amsmath,amssymb,amstext}
\usepackage{babel}
\usepackage{times}
\usepackage[Bjarne]{fncychap}
\usepackage[dontkeepoldnames]{sphinx}

\usepackage{geometry}

% Include hyperref last.
\usepackage{hyperref}
% Fix anchor placement for figures with captions.
\usepackage{hypcap}% it must be loaded after hyperref.
% Set up styles of URL: it should be placed after hyperref.
\urlstyle{same}

\addto\captionsenglish{\renewcommand{\figurename}{Fig.}}
\addto\captionsenglish{\renewcommand{\tablename}{Table}}
\addto\captionsenglish{\renewcommand{\literalblockname}{Listing}}

\addto\captionsenglish{\renewcommand{\literalblockcontinuedname}{continued from previous page}}
\addto\captionsenglish{\renewcommand{\literalblockcontinuesname}{continues on next page}}

\addto\extrasenglish{\def\pageautorefname{page}}





\title{SecretSharingProject Documentation}
\date{May 20, 2018}
\release{0}
\author{Ty Danet, Brandon Artner}
\newcommand{\sphinxlogo}{\vbox{}}
\renewcommand{\releasename}{Release}
\makeindex

\begin{document}

\maketitle
\sphinxtableofcontents
\phantomsection\label{\detokenize{index::doc}}

\index{ThresTree (module)}

\chapter{ThresTree}
\label{\detokenize{index:module-ThresTree}}\label{\detokenize{index:secret-sharing-capstone}}\label{\detokenize{index:threstree}}\index{ThresTree (class in ThresTree)}

\begin{fulllineitems}
\phantomsection\label{\detokenize{index:ThresTree.ThresTree}}\pysigline{\sphinxbfcode{class }\sphinxcode{ThresTree.}\sphinxbfcode{ThresTree}}
Tree designed to be used with Adi Shamir’s (n, k)-thresholding scheme.
\index{root (ThresTree.ThresTree attribute)}

\begin{fulllineitems}
\phantomsection\label{\detokenize{index:ThresTree.ThresTree.root}}\pysigline{\sphinxbfcode{root}}
\sphinxstyleemphasis{TreeNode} \textendash{} The root node of the tree.

\end{fulllineitems}

\index{addChild() (ThresTree.ThresTree method)}

\begin{fulllineitems}
\phantomsection\label{\detokenize{index:ThresTree.ThresTree.addChild}}\pysiglinewithargsret{\sphinxbfcode{addChild}}{\emph{addr}}{}
Add a child node to a node at a given address.
\begin{quote}\begin{description}
\item[{Parameters}] \leavevmode
\sphinxstyleliteralstrong{addr} (\sphinxstyleliteralemphasis{String}) \textendash{} Address of node that new node is being added to with the format ‘\#:\#:\#:…’.

\end{description}\end{quote}

\end{fulllineitems}

\index{propagate() (ThresTree.ThresTree method)}

\begin{fulllineitems}
\phantomsection\label{\detokenize{index:ThresTree.ThresTree.propagate}}\pysiglinewithargsret{\sphinxbfcode{propagate}}{\emph{data=None}}{}
Propogates the tree with data. If none is given a DSA key-pair will be generated by the Voter class.
\begin{quote}\begin{description}
\item[{Parameters}] \leavevmode
\sphinxstyleliteralstrong{data} (\sphinxstyleliteralemphasis{int}) \textendash{} The data to be propogated down the tree. (Usually None)

\end{description}\end{quote}

\end{fulllineitems}

\index{removeChild() (ThresTree.ThresTree method)}

\begin{fulllineitems}
\phantomsection\label{\detokenize{index:ThresTree.ThresTree.removeChild}}\pysiglinewithargsret{\sphinxbfcode{removeChild}}{\emph{addr}}{}
Removes a child node of a node at a given address. Removes a random child node.
\begin{quote}\begin{description}
\item[{Parameters}] \leavevmode
\sphinxstyleliteralstrong{addr} (\sphinxstyleliteralemphasis{String}) \textendash{} Address of node that node is being removed from with the format ‘\#:\#:\#:…’.

\end{description}\end{quote}

\end{fulllineitems}

\index{search() (ThresTree.ThresTree method)}

\begin{fulllineitems}
\phantomsection\label{\detokenize{index:ThresTree.ThresTree.search}}\pysiglinewithargsret{\sphinxbfcode{search}}{\emph{name}}{}
Searches for a node at a given address.
\begin{quote}\begin{description}
\item[{Parameters}] \leavevmode
\sphinxstyleliteralstrong{name} (\sphinxstyleliteralemphasis{String}) \textendash{} Address of node with the format ‘\#:\#:\#:…’.

\item[{Returns}] \leavevmode
A node with the corresponding address.

\item[{Return type}] \leavevmode
{\hyperref[\detokenize{index:ThresTree.TreeNode}]{\sphinxcrossref{TreeNode}}}

\end{description}\end{quote}

\end{fulllineitems}


\end{fulllineitems}

\index{TreeNode (class in ThresTree)}

\begin{fulllineitems}
\phantomsection\label{\detokenize{index:ThresTree.TreeNode}}\pysiglinewithargsret{\sphinxbfcode{class }\sphinxcode{ThresTree.}\sphinxbfcode{TreeNode}}{\emph{addr}, \emph{parent=None}}{}
Nodes used by the Tree object
\index{addr (ThresTree.TreeNode attribute)}

\begin{fulllineitems}
\phantomsection\label{\detokenize{index:ThresTree.TreeNode.addr}}\pysigline{\sphinxbfcode{addr}}
\sphinxstyleemphasis{String} \textendash{} Address of this node.

\end{fulllineitems}

\index{parent (ThresTree.TreeNode attribute)}

\begin{fulllineitems}
\phantomsection\label{\detokenize{index:ThresTree.TreeNode.parent}}\pysigline{\sphinxbfcode{parent}}
\sphinxstyleemphasis{TreeNode} \textendash{} Parent node.

\end{fulllineitems}

\index{documents (ThresTree.TreeNode attribute)}

\begin{fulllineitems}
\phantomsection\label{\detokenize{index:ThresTree.TreeNode.documents}}\pysigline{\sphinxbfcode{documents}}
\sphinxstyleemphasis{list} \textendash{} Documents list.

\end{fulllineitems}

\index{voter (ThresTree.TreeNode attribute)}

\begin{fulllineitems}
\phantomsection\label{\detokenize{index:ThresTree.TreeNode.voter}}\pysigline{\sphinxbfcode{voter}}
\sphinxstyleemphasis{Voter} \textendash{} Voter object.

\end{fulllineitems}

\index{finalized (ThresTree.TreeNode attribute)}

\begin{fulllineitems}
\phantomsection\label{\detokenize{index:ThresTree.TreeNode.finalized}}\pysigline{\sphinxbfcode{finalized}}
\sphinxstyleemphasis{boolean} \textendash{} Keeps track of whether this node has been finalized.

\end{fulllineitems}

\index{current\_vote (ThresTree.TreeNode attribute)}

\begin{fulllineitems}
\phantomsection\label{\detokenize{index:ThresTree.TreeNode.current_vote}}\pysigline{\sphinxbfcode{current\_vote}}
\sphinxstyleemphasis{String, String} \textendash{} The documents that is currently being voted on.

\end{fulllineitems}

\index{already\_voted (ThresTree.TreeNode attribute)}

\begin{fulllineitems}
\phantomsection\label{\detokenize{index:ThresTree.TreeNode.already_voted}}\pysigline{\sphinxbfcode{already\_voted}}
\sphinxstyleemphasis{list} \textendash{} List of node address that have already voted.

\end{fulllineitems}


\begin{sphinxadmonition}{note}{Note:}
already\_voted variable is only used if this node has child nodes.
\end{sphinxadmonition}
\index{finalize() (ThresTree.TreeNode method)}

\begin{fulllineitems}
\phantomsection\label{\detokenize{index:ThresTree.TreeNode.finalize}}\pysiglinewithargsret{\sphinxbfcode{finalize}}{\emph{data=None}}{}
Creates a voter object for each non-leaf node and sets key data for each leaf.
\begin{quote}\begin{description}
\item[{Parameters}] \leavevmode
\sphinxstyleliteralstrong{data} (\sphinxstyleliteralemphasis{int}\sphinxstyleliteralemphasis{, }\sphinxstyleliteralemphasis{int}) \textendash{} key data from parent nodes

\item[{Returns}] \leavevmode
True if the signature is correct, False otherwise.

\item[{Return type}] \leavevmode
boolean

\end{description}\end{quote}

\end{fulllineitems}

\index{show\_documents() (ThresTree.TreeNode method)}

\begin{fulllineitems}
\phantomsection\label{\detokenize{index:ThresTree.TreeNode.show_documents}}\pysiglinewithargsret{\sphinxbfcode{show\_documents}}{\emph{verified\_only=False}}{}
Displays all of the documents for root.
\begin{quote}\begin{description}
\item[{Parameters}] \leavevmode
\sphinxstyleliteralstrong{verified\_only} (\sphinxstyleliteralemphasis{boolean}) \textendash{} Flag to only print verified/signed documents

\end{description}\end{quote}

\begin{sphinxadmonition}{note}{Todo:}
Implement this for all levels of ducument sets. e.g. print documents for address 0:2.
Requires implementing DSA signing on all levels.
\end{sphinxadmonition}

\end{fulllineitems}

\index{split() (ThresTree.TreeNode method)}

\begin{fulllineitems}
\phantomsection\label{\detokenize{index:ThresTree.TreeNode.split}}\pysiglinewithargsret{\sphinxbfcode{split}}{\emph{n}, \emph{k}}{}
Split a node from being a leaf to having children.
\begin{quote}\begin{description}
\item[{Parameters}] \leavevmode\begin{itemize}
\item {} 
\sphinxstyleliteralstrong{n} (\sphinxstyleliteralemphasis{int}) \textendash{} The number of key to generate for this voter.

\item {} 
\sphinxstyleliteralstrong{k} (\sphinxstyleliteralemphasis{int}) \textendash{} The number of vote required to pass a vote.

\end{itemize}

\end{description}\end{quote}

\end{fulllineitems}

\index{vote() (ThresTree.TreeNode method)}

\begin{fulllineitems}
\phantomsection\label{\detokenize{index:ThresTree.TreeNode.vote}}\pysiglinewithargsret{\sphinxbfcode{vote}}{\emph{doc}, \emph{key=None}}{}
Votes on a given document
\begin{quote}\begin{description}
\item[{Parameters}] \leavevmode\begin{itemize}
\item {} 
\sphinxstyleliteralstrong{doc} (\sphinxstyleliteralemphasis{String}\sphinxstyleliteralemphasis{, }\sphinxstyleliteralemphasis{String}) \textendash{} The document being voted on. First element is the file name. Second is the document’s text.

\item {} 
\sphinxstyleliteralstrong{key} (\sphinxstyleliteralemphasis{int}\sphinxstyleliteralemphasis{, }\sphinxstyleliteralemphasis{int}) \textendash{} key data passed up from child nodes

\end{itemize}

\end{description}\end{quote}

\begin{sphinxadmonition}{note}{Todo:}
Turn current\_vote into a queue of documents waiting to be voted on.
\end{sphinxadmonition}

\end{fulllineitems}


\end{fulllineitems}

\phantomsection\label{\detokenize{index:module-Voter}}\index{Voter (module)}

\chapter{Voter}
\label{\detokenize{index:voter}}\index{Voter (class in Voter)}

\begin{fulllineitems}
\phantomsection\label{\detokenize{index:Voter.Voter}}\pysiglinewithargsret{\sphinxbfcode{class }\sphinxcode{Voter.}\sphinxbfcode{Voter}}{\emph{node}, \emph{n}, \emph{k}, \emph{prime\_size=1024}}{}~\begin{description}
\item[{Voter implements a (n,k)-threshold scheme which is capable of submiting}] \leavevmode
a signature.

\end{description}
\index{prime\_size (Voter.Voter attribute)}

\begin{fulllineitems}
\phantomsection\label{\detokenize{index:Voter.Voter.prime_size}}\pysigline{\sphinxbfcode{prime\_size}}
\sphinxstyleemphasis{int} \textendash{} Size of the prime number. (Default: 1024)

\end{fulllineitems}

\index{p (Voter.Voter attribute)}

\begin{fulllineitems}
\phantomsection\label{\detokenize{index:Voter.Voter.p}}\pysigline{\sphinxbfcode{p}}
\sphinxstyleemphasis{int} \textendash{} The prime number.

\end{fulllineitems}

\index{n (Voter.Voter attribute)}

\begin{fulllineitems}
\phantomsection\label{\detokenize{index:Voter.Voter.n}}\pysigline{\sphinxbfcode{n}}
\sphinxstyleemphasis{int} \textendash{} The number of keys to make with Shamir’s Scheme.

\end{fulllineitems}

\index{k (Voter.Voter attribute)}

\begin{fulllineitems}
\phantomsection\label{\detokenize{index:Voter.Voter.k}}\pysigline{\sphinxbfcode{k}}
\sphinxstyleemphasis{int} \textendash{} The number of keys required to construct the data.

\end{fulllineitems}

\index{values (Voter.Voter attribute)}

\begin{fulllineitems}
\phantomsection\label{\detokenize{index:Voter.Voter.values}}\pysigline{\sphinxbfcode{values}}
\sphinxstyleemphasis{list} \textendash{} The values for the active Asynchronous Neville’s Method.

\end{fulllineitems}

\index{pubKey (Voter.Voter attribute)}

\begin{fulllineitems}
\phantomsection\label{\detokenize{index:Voter.Voter.pubKey}}\pysigline{\sphinxbfcode{pubKey}}
\sphinxstyleemphasis{PubKey} \textendash{} Public Key object for DSA.

\end{fulllineitems}

\index{node (Voter.Voter attribute)}

\begin{fulllineitems}
\phantomsection\label{\detokenize{index:Voter.Voter.node}}\pysigline{\sphinxbfcode{node}}
\sphinxstyleemphasis{TreeNoe} \textendash{} The node this Voter belongs to.

\end{fulllineitems}


\begin{sphinxadmonition}{note}{Note:}
Digital Signature Specifications (DSS) specifies that the prime length should be 1024/2048 (Can’t Remember).
\end{sphinxadmonition}
\index{add\_key\_to\_signature() (Voter.Voter method)}

\begin{fulllineitems}
\phantomsection\label{\detokenize{index:Voter.Voter.add_key_to_signature}}\pysiglinewithargsret{\sphinxbfcode{add\_key\_to\_signature}}{\emph{key}, \emph{doc}}{}
Function implements the Asynchronous Neville’s Method and adds the given key to the in progress vote.
\begin{quote}\begin{description}
\item[{Parameters}] \leavevmode\begin{itemize}
\item {} 
\sphinxstyleliteralstrong{key} (\sphinxstyleliteralemphasis{int}\sphinxstyleliteralemphasis{, }\sphinxstyleliteralemphasis{int}) \textendash{} The key being added to the LIP.

\item {} 
\sphinxstyleliteralstrong{doc} \textendash{} The document being voted on.

\end{itemize}

\end{description}\end{quote}

\begin{sphinxadmonition}{note}{Todo:}
Ensure that the submitted key is actually valid. ie. It is from the set of keys originally generated
for this voter.
\end{sphinxadmonition}

\begin{sphinxadmonition}{note}{Note:}
This function returns both a signature if there is one, and a boolean of whether
or not a vote has finished.
\end{sphinxadmonition}

\end{fulllineitems}

\index{generate\_scheme() (Voter.Voter method)}

\begin{fulllineitems}
\phantomsection\label{\detokenize{index:Voter.Voter.generate_scheme}}\pysiglinewithargsret{\sphinxbfcode{generate\_scheme}}{\emph{data=None}}{}
Generates the private-public key pair, the polynomial, and the keys for this voter.
\begin{quote}\begin{description}
\item[{Parameters}] \leavevmode
\sphinxstyleliteralstrong{data} \textendash{} The data to be used as the secret data.
(Note: None means the voter is for root node.)

\end{description}\end{quote}

\begin{sphinxadmonition}{note}{Todo:}
Implement non-root DSA key pairs. This is to allow for signing of sub-root
document sets. Problem consists of being given a private key and having to
generate a public key.
\end{sphinxadmonition}

\end{fulllineitems}

\index{sign() (Voter.Voter method)}

\begin{fulllineitems}
\phantomsection\label{\detokenize{index:Voter.Voter.sign}}\pysiglinewithargsret{\sphinxbfcode{sign}}{\emph{doc}, \emph{private\_key}}{}
Function creates the digital signature for a given document with this voter’s private key.
\begin{quote}\begin{description}
\item[{Parameters}] \leavevmode
\sphinxstyleliteralstrong{doc} \textendash{} The document that is being signed.

\item[{Returns}] \leavevmode
A signature if this node has a public key, else None.

\item[{Return type}] \leavevmode
(int, int)

\end{description}\end{quote}

\begin{sphinxadmonition}{note}{Todo:}
Right now the doc param is a string 2-tuple, this might be better as just a String since all
that is needed is the document text and it would be more intuitive for future users.
\end{sphinxadmonition}

\end{fulllineitems}

\index{verify() (Voter.Voter method)}

\begin{fulllineitems}
\phantomsection\label{\detokenize{index:Voter.Voter.verify}}\pysiglinewithargsret{\sphinxbfcode{verify}}{\emph{doc}, \emph{signature}}{}
Function verifies a digital signature. for a given document with this voter’s public key.
\begin{quote}\begin{description}
\item[{Parameters}] \leavevmode\begin{itemize}
\item {} 
\sphinxstyleliteralstrong{doc} (\sphinxstyleliteralemphasis{String}) \textendash{} The document that is being signed.

\item {} 
\sphinxstyleliteralstrong{signature} (\sphinxstyleliteralemphasis{int}\sphinxstyleliteralemphasis{, }\sphinxstyleliteralemphasis{int}) \textendash{} DSA signature being verified.

\end{itemize}

\item[{Returns}] \leavevmode
True if the voter has a public key and the signature is correct, False otherwise.

\item[{Return type}] \leavevmode
boolean

\end{description}\end{quote}

\end{fulllineitems}


\end{fulllineitems}

\phantomsection\label{\detokenize{index:module-TreeMaker}}\index{TreeMaker (module)}

\chapter{TreeMaker}
\label{\detokenize{index:treemaker}}\index{TreeMaker (class in TreeMaker)}

\begin{fulllineitems}
\phantomsection\label{\detokenize{index:TreeMaker.TreeMaker}}\pysigline{\sphinxbfcode{class }\sphinxcode{TreeMaker.}\sphinxbfcode{TreeMaker}}
docstring for TreeMaker
\index{help() (TreeMaker.TreeMaker method)}

\begin{fulllineitems}
\phantomsection\label{\detokenize{index:TreeMaker.TreeMaker.help}}\pysiglinewithargsret{\sphinxbfcode{help}}{\emph{command=None}}{}
Prints help information.

Args:
command (String): The command to print help information for.

\begin{sphinxadmonition}{note}{Todo:}
The help documentation needs to finished.
\end{sphinxadmonition}

\end{fulllineitems}

\index{parse() (TreeMaker.TreeMaker method)}

\begin{fulllineitems}
\phantomsection\label{\detokenize{index:TreeMaker.TreeMaker.parse}}\pysiglinewithargsret{\sphinxbfcode{parse}}{\emph{command}}{}
Parses user inputted commands.
\begin{quote}\begin{description}
\item[{Parameters}] \leavevmode
\sphinxstyleliteralstrong{user\_input} (\sphinxstyleliteralemphasis{String}) \textendash{} The User input to be parsed.

\end{description}\end{quote}

\end{fulllineitems}

\index{repl() (TreeMaker.TreeMaker method)}

\begin{fulllineitems}
\phantomsection\label{\detokenize{index:TreeMaker.TreeMaker.repl}}\pysiglinewithargsret{\sphinxbfcode{repl}}{}{}
REPL function for tree maker. Loops until quit is called.

\end{fulllineitems}


\end{fulllineitems}

\phantomsection\label{\detokenize{index:module-VoteSession}}\index{VoteSession (module)}

\chapter{VoteSession}
\label{\detokenize{index:votesession}}\index{VoteSim (class in VoteSession)}

\begin{fulllineitems}
\phantomsection\label{\detokenize{index:VoteSession.VoteSim}}\pysiglinewithargsret{\sphinxbfcode{class }\sphinxcode{VoteSession.}\sphinxbfcode{VoteSim}}{\emph{tree}}{}
docstring for VoteSim
\index{help() (VoteSession.VoteSim method)}

\begin{fulllineitems}
\phantomsection\label{\detokenize{index:VoteSession.VoteSim.help}}\pysiglinewithargsret{\sphinxbfcode{help}}{\emph{command=None}}{}
Prints help information.
\begin{quote}\begin{description}
\item[{Parameters}] \leavevmode\begin{itemize}
\item {} 
\sphinxstyleliteralstrong{command} (\sphinxstyleliteralemphasis{String}) \textendash{} The command to print help information for.

\item {} 
\sphinxstyleliteralstrong{Todo} \textendash{} The help documentation needs to finished.

\end{itemize}

\end{description}\end{quote}

\end{fulllineitems}

\index{parse() (VoteSession.VoteSim method)}

\begin{fulllineitems}
\phantomsection\label{\detokenize{index:VoteSession.VoteSim.parse}}\pysiglinewithargsret{\sphinxbfcode{parse}}{\emph{user\_input}}{}
Parses user inputted commands.
\begin{quote}\begin{description}
\item[{Parameters}] \leavevmode\begin{itemize}
\item {} 
\sphinxstyleliteralstrong{user\_input} (\sphinxstyleliteralemphasis{String}) \textendash{} The User input to be parsed.

\item {} 
\sphinxstyleliteralstrong{Todo} \textendash{} Input verification needs to be improved.

\end{itemize}

\end{description}\end{quote}

\end{fulllineitems}

\index{repl() (VoteSession.VoteSim method)}

\begin{fulllineitems}
\phantomsection\label{\detokenize{index:VoteSession.VoteSim.repl}}\pysiglinewithargsret{\sphinxbfcode{repl}}{}{}
REPL function to run a voting session. Loops until quit is called.

\end{fulllineitems}


\end{fulllineitems}

\phantomsection\label{\detokenize{index:module-Toolkit}}\index{Toolkit (module)}

\chapter{Toolkit}
\label{\detokenize{index:toolkit}}\index{Polynomial (class in Toolkit)}

\begin{fulllineitems}
\phantomsection\label{\detokenize{index:Toolkit.Polynomial}}\pysiglinewithargsret{\sphinxbfcode{class }\sphinxcode{Toolkit.}\sphinxbfcode{Polynomial}}{\emph{coefficients}}{}
Class to evaluate simulated polynomials

\end{fulllineitems}

\index{convvert\_to\_PEM() (in module Toolkit)}

\begin{fulllineitems}
\phantomsection\label{\detokenize{index:Toolkit.convvert_to_PEM}}\pysiglinewithargsret{\sphinxcode{Toolkit.}\sphinxbfcode{convvert\_to\_PEM}}{\emph{private\_key}}{}
Converts a private key integer to PEM format

\begin{sphinxadmonition}{note}{Todo:}
Implement this… Maybe… This may be needed for making document sets on lower levels of the hierarchy.
\end{sphinxadmonition}

\end{fulllineitems}

\index{data\_to\_key() (in module Toolkit)}

\begin{fulllineitems}
\phantomsection\label{\detokenize{index:Toolkit.data_to_key}}\pysiglinewithargsret{\sphinxcode{Toolkit.}\sphinxbfcode{data\_to\_key}}{\emph{data}, \emph{p}}{}
Calculates a congruent integer to the numerator that is divisible by the denominator.
\begin{quote}\begin{description}
\item[{Parameters}] \leavevmode\begin{itemize}
\item {} 
\sphinxstyleliteralstrong{data} (\sphinxstyleliteralemphasis{int}) \textendash{} 
.


\item {} 
\sphinxstyleliteralstrong{n} (\sphinxstyleliteralemphasis{int}) \textendash{} The maximum number the first term in the key can be.

\end{itemize}

\item[{Returns}] \leavevmode
key. The key pair from the given encoded integer.

\item[{Return type}] \leavevmode
(int, int)

\end{description}\end{quote}

\end{fulllineitems}

\index{find\_divisible\_congruency() (in module Toolkit)}

\begin{fulllineitems}
\phantomsection\label{\detokenize{index:Toolkit.find_divisible_congruency}}\pysiglinewithargsret{\sphinxcode{Toolkit.}\sphinxbfcode{find\_divisible\_congruency}}{\emph{fraction}, \emph{p}}{}
Calculates a congruent integer to the numerator that is divisible by the denominator.
\begin{quote}\begin{description}
\item[{Parameters}] \leavevmode
\sphinxstyleliteralstrong{fraction} (\sphinxstyleliteralemphasis{Fraction}) \textendash{} fraction, the fraction in question.

\item[{Returns}] \leavevmode
The z such that z = (num + i*prime)/den and z is an integer (without trunction).

\item[{Return type}] \leavevmode
int

\end{description}\end{quote}

\end{fulllineitems}

\index{generate\_keys() (in module Toolkit)}

\begin{fulllineitems}
\phantomsection\label{\detokenize{index:Toolkit.generate_keys}}\pysiglinewithargsret{\sphinxcode{Toolkit.}\sphinxbfcode{generate\_keys}}{\emph{poly}, \emph{n}, \emph{p}}{}
Creates the keys for distributing.
\begin{quote}\begin{description}
\item[{Parameters}] \leavevmode\begin{itemize}
\item {} 
\sphinxstyleliteralstrong{poly} ({\hyperref[\detokenize{index:Toolkit.Polynomial}]{\sphinxcrossref{\sphinxstyleliteralemphasis{Polynomial}}}}) \textendash{} Polynomial function to create the keys.

\item {} 
\sphinxstyleliteralstrong{n} (\sphinxstyleliteralemphasis{int}) \textendash{} 

\item {} 
\sphinxstyleliteralstrong{p} (\sphinxstyleliteralemphasis{int}) \textendash{} The prime modulus.

\end{itemize}

\item[{Returns}] \leavevmode
list of key pairs generated by Shamir’s Scheme.

\item[{Return type}] \leavevmode
list of int tuples

\end{description}\end{quote}

\end{fulllineitems}

\index{generate\_polynomial() (in module Toolkit)}

\begin{fulllineitems}
\phantomsection\label{\detokenize{index:Toolkit.generate_polynomial}}\pysiglinewithargsret{\sphinxcode{Toolkit.}\sphinxbfcode{generate\_polynomial}}{\emph{data}, \emph{k}, \emph{p}}{}
Makes a random polynomial with data as the coeficient term.
\begin{quote}\begin{description}
\item[{Parameters}] \leavevmode\begin{itemize}
\item {} 
\sphinxstyleliteralstrong{data} (\sphinxstyleliteralemphasis{int}) \textendash{} The data as a numeric value to be split.

\item {} 
\sphinxstyleliteralstrong{k} (\sphinxstyleliteralemphasis{int}) \textendash{} The number of points required to interpolate the polynomial,
which is the polynomial’s order plus one.

\item {} 
\sphinxstyleliteralstrong{p} (\sphinxstyleliteralemphasis{int}) \textendash{} The prime modulus.

\end{itemize}

\item[{Returns}] \leavevmode
polynomial as a lambda function.

\item[{Return type}] \leavevmode
{\hyperref[\detokenize{index:Toolkit.Polynomial}]{\sphinxcrossref{Polynomial}}}

\end{description}\end{quote}

\end{fulllineitems}

\index{key\_to\_data() (in module Toolkit)}

\begin{fulllineitems}
\phantomsection\label{\detokenize{index:Toolkit.key_to_data}}\pysiglinewithargsret{\sphinxcode{Toolkit.}\sphinxbfcode{key\_to\_data}}{\emph{key}, \emph{n}}{}
Calculates a congruent integer to the numerator that is divisible by the denominator.
\begin{quote}\begin{description}
\item[{Parameters}] \leavevmode\begin{itemize}
\item {} 
\sphinxstyleliteralstrong{key} (\sphinxstyleliteralemphasis{int}\sphinxstyleliteralemphasis{, }\sphinxstyleliteralemphasis{int}) \textendash{} The key pair to be encoded.

\item {} 
\sphinxstyleliteralstrong{n} (\sphinxstyleliteralemphasis{int}) \textendash{} The maximum number the first term in the key can be.

\end{itemize}

\item[{Returns}] \leavevmode
data. The key pair encoded as an integer.

\item[{Return type}] \leavevmode
int

\end{description}\end{quote}

\end{fulllineitems}

\index{next\_multiple\_of\_128() (in module Toolkit)}

\begin{fulllineitems}
\phantomsection\label{\detokenize{index:Toolkit.next_multiple_of_128}}\pysiglinewithargsret{\sphinxcode{Toolkit.}\sphinxbfcode{next\_multiple\_of\_128}}{\emph{data}}{}
Calculates the smallest number of bits bigger than the length of data and a multiple of 128.

\end{fulllineitems}

\index{nthOrderLag() (in module Toolkit)}

\begin{fulllineitems}
\phantomsection\label{\detokenize{index:Toolkit.nthOrderLag}}\pysiglinewithargsret{\sphinxcode{Toolkit.}\sphinxbfcode{nthOrderLag}}{\emph{values}, \emph{n}}{}
Calculates the (n)th-ordered Lagrangian Interpolating Polynomial, evaluated at 0.
\begin{quote}\begin{description}
\item[{Parameters}] \leavevmode
\sphinxstyleliteralstrong{n} (\sphinxstyleliteralemphasis{int}) \textendash{} The order of the next LIP.

\item[{Returns}] \leavevmode
A fraction containing the value of the nth LIP, evaluated at 0.

\item[{Return type}] \leavevmode
Fraction

\end{description}\end{quote}

\end{fulllineitems}

\index{random\_distinct() (in module Toolkit)}

\begin{fulllineitems}
\phantomsection\label{\detokenize{index:Toolkit.random_distinct}}\pysiglinewithargsret{\sphinxcode{Toolkit.}\sphinxbfcode{random\_distinct}}{\emph{lower\_bound}, \emph{upper\_bound}, \emph{size}}{}
Generates lists of distinct random numbers,
\begin{quote}\begin{description}
\item[{Parameters}] \leavevmode\begin{itemize}
\item {} 
\sphinxstyleliteralstrong{lower\_bound} (\sphinxstyleliteralemphasis{int}) \textendash{} Low bound for searching.

\item {} 
\sphinxstyleliteralstrong{upper\_bound} (\sphinxstyleliteralemphasis{int}) \textendash{} High bound for searching.

\item {} 
\sphinxstyleliteralstrong{size} (\sphinxstyleliteralemphasis{int}) \textendash{} Length of list of numbers.

\end{itemize}

\item[{Returns}] \leavevmode
List of random numbers within given bounds.

\item[{Return type}] \leavevmode
int

\end{description}\end{quote}

\end{fulllineitems}



\renewcommand{\indexname}{Python Module Index}
\begin{sphinxtheindex}
\def\bigletter#1{{\Large\sffamily#1}\nopagebreak\vspace{1mm}}
\bigletter{t}
\item {\sphinxstyleindexentry{ThresTree}}\sphinxstyleindexpageref{index:\detokenize{module-ThresTree}}
\item {\sphinxstyleindexentry{Toolkit}}\sphinxstyleindexpageref{index:\detokenize{module-Toolkit}}
\item {\sphinxstyleindexentry{TreeMaker}}\sphinxstyleindexpageref{index:\detokenize{module-TreeMaker}}
\indexspace
\bigletter{v}
\item {\sphinxstyleindexentry{Voter}}\sphinxstyleindexpageref{index:\detokenize{module-Voter}}
\item {\sphinxstyleindexentry{VoteSession}}\sphinxstyleindexpageref{index:\detokenize{module-VoteSession}}
\end{sphinxtheindex}

\renewcommand{\indexname}{Index}
\printindex
\end{document}